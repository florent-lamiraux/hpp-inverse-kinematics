\documentclass{article}

\usepackage{graphicx}
\usepackage{amssymb}
\usepackage{amsmath}
\usepackage{cancel}

\newcommand\linvel{\mathbf{v}}
\newcommand\trans{\mathbf{t}}
\newcommand\conf{\mathbf{q}}
\newcommand\reals{\mathbb{R}}

\title{Computation of the exact inverse kinematics solution of a 6D robotic arm}
\author{Florent Lamiraux - CNRS}

\begin{document}
\maketitle

\begin{figure}
  \begin{center}
    \def\svgwidth {.6\linewidth}
    \graphicspath{{./figures/}}
    \input{figures/kinematic-chain.pdf_tex}
  \end{center}
  \caption{Input (blue) and output (red) variables of the explicit constraint that computes the robot arm configuration with respect to the handle pose.}
  \label{fig:kinematic-chain}
\end{figure}

\section{Introduction}

This document explains how to implement exact inverse kinematics of a Stäubli robotic arm as an
explicit constraint in Humanoid Path Planner software. Notation and definitions are the same as
in~\cite{lamiraux:hal-02995125}.

\section{Notation and Definitions}

The constraint is defined as a 6 dimensional \textit{grasp} constraint between a \textit{gripper} and a
\textit{handle}. The \textit{gripper} is attached to the robotic arm end-effector (\texttt{joint1}) and the \textit{handle} is attached to \texttt{joint2} on the composite kinematic chain. \texttt{root} is the joint that holds the robot arm or the global frame (\texttt{"universe"} in \texttt{pinocchio} software).

We denote by
\begin{itemize}
\item $\conf_{in}$ the input variables of the explicit constraint,
\item $\conf_{out}$ the output variables of the explicit constraint,
\item $^0M_1$ the pose of \texttt{joint1} in the world frame,
\item $^0M_2$ the pose of \texttt{joint2} in the world frame,
\item $^0M_r$ the pose of \texttt{root} in the world frame,
\item $^2M_h$ the pose of the handle in \texttt{joint2} frame,
\item $^1M_g$ the pose of the gripper in \texttt{joint1} frame,
\item $^rM_b$ the pose of the robot arm origin (\texttt{base\_link} in URDF
  description) in the \texttt{root} frame.
\end{itemize}

\section{Inverse kinematics}

Exact inverse kinematics computes the 6 joint values of the robotic arm with
respect to the input configuration variables:
\begin{equation}\label{eq:inverse-kinematics}
  \conf_{out} = f(\conf_{in})
\end{equation}
Note that the input variables include a extra degree of freedom that is interpreted as an integer to select among the various solutions of the inverse kinematics.



Homogeneous transformation matrix  $T_{i-1,i}$  from frame $i-1$  to frame $i$  :

\[
\begin{array}{cc}
T_{0,1} =
\begin{bmatrix}
  \cos(q_1) & -\sin(q_1) & 0 & 0 \\
  \sin(q_1) & \cos(q_1) & 0 & 0 \\
  0 & 0 & 1 & r_1 \\
  0 & 0 & 0 & 1 \\

\end{bmatrix}

&

T_{1,2} =
\begin{bmatrix}
  \cos(q_2) & 0 & \sin(q_2) & r_2 \\
  0 & 1 & 0 & 0 \\
  -\sin(q_2) & 0 & \cos(q_2) & 0 \\
  0 & 0 & 0 & 1 \\

\end{bmatrix}
\end{array}
\]

\[
\begin{array}{cc}
T_{2,3} =
\begin{bmatrix}
  \cos(q_3) & 0 & \sin(q_3) & 0 \\
  0 & 1 & 0 & r_3 \\
  -\sin(q_3) & 0 & \cos(q_3) & r_4 \\
  0 & 0 & 0 & 1 \\

\end{bmatrix}
&
T_{3,4} =
\begin{bmatrix}
  \cos(q_4) & -\sin(q_4) & 0 & 0 \\
  \sin(q_4) & \cos(q_4) & 0 & 0 \\
  0 & 0 & 1 & 0 \\
  0 & 0 & 0 & 1 \\

\end{bmatrix}
\end{array}
\]
\[
\begin{array}{cc}
T_{4,5} =
\begin{bmatrix}
  \cos(q_5) & 0 & \sin(q_5) & 0 \\
  0 & 1 & 0 & 0 \\
  -\sin(q_5) & 0 & \cos(q_5) & r_5 \\
  0 & 0 & 0 & 1 \\

\end{bmatrix}

&

T_{5,6} =
\begin{bmatrix}
  \cos(q_6) & -\sin(q_6) & 0 & 0 \\
  \sin(q_6) & \cos(q_6) & 0 & 0\\
  0 & 0 & 1 & r_6\\
  0 & 0 & 0 & 1\\

\end{bmatrix}
\end{array}
\]


Calculation of  $T_{0,5}$ and $T_{3,6}$ to determinate each configuration expressions. We assume that $sin(q_i)=s_i$  for each i between 1 and 6. \\
\begin{align*}
T_{0,3} = T_{0,1} T_{1,2} T_{2,3} T_{3,4} T_{4,5} =
\begin{bmatrix}
  c_1c_{2+3} & -s_1 & c_1s_{2+3} &r_5c_1s_{2+3}-r_3s_1+r_4c_1s_2+c_1r_2\\
  s_1c_{2+3} & c_1 & s_1c_{2+3} & r_5s_1s_{2+3}+r_3c_1+r_4s_1s_2+r_2s_1 \\
  -s_{2+3} & 0 & c_{2+3} & r_5c_{2+3}+r_4c_2+r_1 \\
  0 & 0 & 0 & 1 \\
\end{bmatrix}
  /!\ la matrice de rotation est pas bonne faut la corriger.
\end{align*}

\[
T_{3,6} = T_{3,4} T_{4,5} T_{5,6}= \\
\begin{bmatrix}
  c_4c_5c_6+s_4s_6 & -c_4c_5s_6-s_4c_6 & c_4s_5 & r_6c_4s_5 \\
  s_4c_5c_6-c_4s_6 & -s_4c_5s_6+c_4c_6 & s_4s_5 & r_6s_4s_5 \\
  -s_5c_6 & s_5s_6 & c_5 & r_6c_5 + r_5 \\
  0 & 0 & 0 & 1 \\
\end{bmatrix}
\]

In order to know the configuration for the position of the concourant point, we need to use the translation vector in $T_{0,5}$ matrix and trigonometrix tools.

For $q_1$, with a manipulation of the plane described by the two other articulation we find:

\begin{equation}
  \begin{bmatrix}
    c1 & -s1\\
    s_1 & c_1\\
  \end{bmatrix}
  \begin{bmatrix}
    x\\
    r_3\\
  \end{bmatrix}
  =
  \begin{bmatrix}
    c_1x-s_1r_3\\
    s_1x+c_1r_3\\
  \end{bmatrix}
\end{equation}

we identify with translation matrix from $T_{0,5}$:
\begin{align*}
  c_1x-\cancel{s_1r_3}&=r_5c_1s_{2+3}-\cancel{s_1r_3} +c_1s_2r_4+ c_1r_2\\
  s_1x+\cancel{c_1r_3}&=r_5s_1s_{2+3}+\cancel{c_1r_3}+s_1s_2r_4+s_1r_2\\
\end{align*}
we simplify by $c_1$ and $s_1$ :
\begin{equation}
  x=r_5s_{2+3}+s_2r_4+r_2
\end{equation}

the x we use correspond to the horizontal displacement in the plane of rotation $q_1$, so we want to have x et y from the root frame :
\begin{align*}
  x'&=c_1x\\
  y'&=s_1x\\
\end{align*}

We finaly have :
\begin{align*}
  x'&=c_1(r_5s_{2+3}+s_2r_4+r_2)\\
  y'&=s_1(r_5s_{2+3}+s_2r_4+r_2)\\
\end{align*}

$q_1$ is given by the following expression :
\[
q_1=\text{atan2}(y',x')
\]

on remarque que une autre solution possible est:
\[
q_1=\pi-\text{atan2}(x,y)
\]


For $q_2$ and $q_3$, we consider the plane of rotation defined by $q_1$.
We pose $z'=z-r_1$.
With al-kashi formula we can write that :
\begin{align*}
  \sqrt{x^2+z'^2}^2&=r_4^2+r_5^2-2r_4r_5\cos(\pi-q_3)\\
  x²+z'²&=r_4^2+r_5^2+2r_4r_5cos(q_3)\\
  \cos(q_3)&= \frac{x^2+y'^2-r_4^2-r_5^2}{2r_4r_5}\\
  q_3&=\arccos(\frac{x^2+(y-r_1)^2-r_4^2-r_5^2}{2r_4r_5})\\
\end{align*}
for $q_2$:
\begin{align*}
  x&=\sin(q_2)r_4+r_5\sin(q_2+q_3)\\
  y&=\cos(q_2)r_4+\cos(q_2+q_3)r_5 + r_1\\
\end{align*}
\begin{align*}
  x&=\sin(q_2)r_4+r_5(\sin(q_2)\cos(q_3)+\sin(q_3)\cos(q_2))\\
  y'&=\cos(q_2)r_4+r_5(\cos(q_3)\cos(q_2)-\sin(q_3)\sin(q_2))
\end{align*}

we define:
\begin{align*}
  A&= r_4+\cos(q_3)r_5\\
  B&= \sin(q_3)r_5 \\
  R&=\sqrt{A^2+B^2}\\
  \phi&=\text{atan2}(B,A)\\
\end{align*}

Using the auxiliary angle technique (à preciser?) we find :

\begin{align*}
  x&=R\sin(q_2+\phi)\\
  q_2&=\arcsin(\frac{x}{R})-\phi\\
  \text{or}\\
  q_2&=\pi-\arcsin(\frac{x}{R})-\phi\\
\end{align*}





\section{Jacobian}

In order to implement exact inverse kinematics as an explicit constraint, we need to compute the Jacobian of $f$. For that, let us consider of motion of the kinematic chain that keeps the gripper and handle in the same pose:
\begin{equation}\label{eq:jac1}
\forall t\in\reals,\ ^0M_2(t)\;^2M_h = \;^0M_r(t) \;^rM_1(t)\;^1M_g
\end{equation}
Moreover
$$
\left(\begin{array}{c}
  \;^r\linvel_{1/r} \\ \;^r\omega_{1/r}
\end{array}\right) =
J_{out} \dot{\conf}_{out}
$$
Using homogeneous matrix notation and derivating with respect to time, Equation~(\ref{eq:jac1}) can be written as $\forall t\in\reals$,
\begin{align}\label{eq:jac2}
  ^0M_2
  \left(\begin{array}{ll}[\;^2\omega_{2/0}]_{\times} & \;^2\linvel_{2/0} \\ 0&0\end{array}\right)\;^2M_h =&
    \;^0M_r\left(\begin{array}{ll}[\;^r\omega_{r/0}]_{\times} & \;^r\linvel_{r/0} \\ 0&0\end{array}\right) \;^rM_1\;^1M_g \\
    & + \;^0M_r \;^rM_1\left(\begin{array}{ll}[\;^1\omega_{1/r}]_{\times} & \;^1\linvel_{1/r} \\ 0&0\end{array}\right)\;^1M_g &
\end{align}
\begin{align*}
  \left(\begin{array}{ll}^0R_2[^2\omega_{2/0}]_{\times} & ^0R_2\;^2\linvel_{2/0} \\ 0&0\end{array}\right)\;^2M_h =&
    \left(\begin{array}{ll}^0R_r[^r\omega_{r/0}]_{\times} & ^0R_r\;^r\linvel_{r/0} \\ 0&0\end{array}\right) \;^rM_1\;^1M_g \\
    & + \;^0M_r \left(\begin{array}{ll}^rR_1[^1\omega_{1/r}]_{\times} & ^rR_1\;^1\linvel_{1/r} \\ 0&0\end{array}\right)\;^1M_g &
\end{align*}
%% \begin{align*}
%%   \left(\begin{array}{ll}^0R_2[^2\omega_{2/0}]_{\times} & ^0R_2\;^2\linvel_{2/0} \\ 0&0\end{array}\right)\left(\begin{array}{ll}\;^2R_h & \;^2\trans_h\\ 0&1\end{array}\right) =&
%%     \left(\begin{array}{ll}^0R_r[^r\omega_{r/0}]_{\times} & ^0R_r\;^r\linvel_{r/0} \\ 0&0\end{array}\right) \left(\begin{array}{ll}^rR_g &\;^r\trans_g\\0&1\end{array}\right) \\
%%     & + \left(\begin{array}{ll}^0R_r &\;^0\trans_r\\0&1\end{array}\right) \left(\begin{array}{ll}^rR_1[^1\omega_{1/r}]_{\times} & ^rR_1\;^1\linvel_{1/r} \\ 0&0\end{array}\right)\left(\begin{array}{ll}^1R_g &\;^1\trans_g\\0&1\end{array}\right) &
%% \end{align*}
%% \begin{align*}
%%   \left(\begin{array}{ll}^0R_2[^2\omega_{2/0}]_{\times}\;^2R_h & ^0R_2[^2\omega_{2/0}]_{\times}\;^2\trans_h + ^0R_2\;^2\linvel_{2/0} \\ 0&0\end{array}\right) =
%%     \left(\begin{array}{ll}^0R_r[^r\omega_{r/0}]_{\times}\;^rR_g & ^0R_r[^r\omega_{r/0}]_{\times}\;^r\trans_g + ^0R_r\;^r\linvel_{r/0} \\ 0&0\end{array}\right) \\
%%       + \left(\begin{array}{ll}^0R_r\;^rR_1[^1\omega_{1/r}]_{\times} & ^0R_r\;^rR_1\;^1\linvel_{1/r} \\ 0&0\end{array}\right)\left(\begin{array}{ll}^1R_g &\;^1\trans_g\\0&1\end{array}\right) &
%% \end{align*}
\begin{align*}
  \left(\begin{array}{ll}^0R_2[^2\omega_{2/0}]_{\times}\;^2R_h & ^0R_2[^2\omega_{2/0}]_{\times}\;^2\trans_h + ^0R_2\;^2\linvel_{2/0} \\ 0&0\end{array}\right) =
    \left(\begin{array}{ll}^0R_r[^r\omega_{r/0}]_{\times}\;^rR_g & ^0R_r[^r\omega_{r/0}]_{\times}\;^r\trans_g + ^0R_r\;^r\linvel_{r/0} \\ 0&0\end{array}\right) \\
      + \left(\begin{array}{ll}^0R_r\;^rR_1[^1\omega_{1/r}]_{\times}\;^1R_g &
        ^0R_r\;^rR_1[^1\omega_{1/r}]_{\times}\;^1\trans_g  + ^0R_r\;^rR_1\;^1\linvel_{1/r}\\ 0&0\end{array}\right)&
\end{align*}
Extracting the upper blocks of this matrix equality, we get
\begin{align*}
  ^0R_2[^2\omega_{2/0}]_{\times}\;^2R_h &= ^0R_r[^r\omega_{r/0}]_{\times}\;^rR_g + ^0R_r\;^rR_1[^1\omega_{1/r}]_{\times}\;^1R_g \\
  ^0R_2[^2\omega_{2/0}]_{\times}\;^2\trans_h + ^0R_2\;^2\linvel_{2/0} &=
  ^0R_r[^r\omega_{r/0}]_{\times}\;^r\trans_g + ^0R_r\;^r\linvel_{r/0} + ^0R_r\;^rR_1[^1\omega_{1/r}]_{\times}\;^1\trans_g  + ^0R_r\;^rR_1\;^1\linvel_{1/r} \\
  ^0R_2[^2\omega_{2/0}]_{\times}\;^2R_h &= ^0R_r[^r\omega_{r/0}]_{\times}\;^rR_g + ^0R_1[^1\omega_{1/r}]_{\times}\;^1R_g \\
  ^0R_2[^2\omega_{2/0}]_{\times}\;^2\trans_h + ^0R_2\;^2\linvel_{2/0} &=
  ^0R_r[^r\omega_{r/0}]_{\times}\;^r\trans_g + ^0R_r\;^r\linvel_{r/0} + ^0R_1[^1\omega_{1/r}]_{\times}\;^1\trans_g  + ^0R_1\;^1\linvel_{1/r} \\
  [^0\omega_{2/0}]_{\times}\;^0R_h &= [^0\omega_{r/0}]_{\times}\;^0R_g + [^0\omega_{1/r}]_{\times}\;^0R_g \\
  ^0R_2[^2\omega_{2/0}]_{\times}\;^2\trans_h + ^0R_2\;^2\linvel_{2/0} &=
  ^0R_r[^r\omega_{r/0}]_{\times}\;^r\trans_g + ^0R_r\;^r\linvel_{r/0} + ^0R_1[^1\omega_{1/r}]_{\times}\;^1\trans_g  + ^0R_1\;^1\linvel_{1/r} \\
\end{align*}
\bibliographystyle{plain}
\bibliography{inverse-kinematics}

\end{document}
