\documentclass{article}

\usepackage{graphicx}
\usepackage{amssymb}
\usepackage{amsmath}
\usepackage{cancel}

\newcommand\linvel{\mathbf{v}}
\newcommand\trans{\mathbf{t}}
\newcommand\conf{\mathbf{q}}
\newcommand\reals{\mathbb{R}}

\title{Computation of the exact inverse kinematics solution of a 6D robotic arm}
\author{Florent Lamiraux - CNRS}

\begin{document}
\maketitle

\begin{figure}
  \begin{center}
    \def\svgwidth {.6\linewidth}
    \graphicspath{{./figures/}}
    \input{figures/kinematic-chain.pdf_tex}
  \end{center}
  \caption{Input (blue) and output (red) variables of the explicit constraint that computes the robot arm configuration with respect to the handle pose.}
  \label{fig:kinematic-chain}
\end{figure}

\section{Introduction}

This document explains how to implement exact inverse kinematics of a Stäubli robotic arm as an
explicit constraint in Humanoid Path Planner software. Notation and definitions are the same as
in~\cite{lamiraux:hal-02995125}.

\section{Notation and Definitions}

The constraint is defined as a 6 dimensional \textit{grasp} constraint between a \textit{gripper} and a
\textit{handle}. The \textit{gripper} is attached to the robotic arm end-effector (\texttt{joint a}) and the \textit{handle} is attached to \texttt{joint b} on the composite kinematic chain. \texttt{root} is the joint that holds the robot arm or the global frame (\texttt{"universe"} in \texttt{pinocchio} software).

We denote by
\begin{itemize}
\item $\conf_{in}$ the input variables of the explicit constraint,
\item $\conf_{out}$ the output variables of the explicit constraint,
\item $^0M_a$ the pose of \texttt{joint a} in the world frame,
\item $^0M_b$ the pose of \texttt{joint b} in the world frame,
\item $^0M_r$ the pose of \texttt{root} in the world frame,
\item $^bM_h$ the pose of the handle in \texttt{joint b} frame,
\item $^aM_g$ the pose of the gripper in \texttt{joint a} frame,
\item $^0M_h$ the pose of the handle in the world frame,
\item $^0M_g$ the pose of the gripper in world frame,
\item $^rM_{bl}$ the pose of the robot arm origin (\texttt{base\_link} in URDF
  description) in the \texttt{root} frame.
\end{itemize}

\section{Inverse kinematics}

\subsection{Implicit formulation}

The implicit formulation is represented by a function with values in $\reals^3\times SO(3)$:
\begin{equation}\label{eq:implicit-formulation}
^0M_g^{-1}\;^0M_h(\conf) = rhs
\end{equation}
where $rhs\in\reals^3\times SO(3)$ is the right hand side of the constraint.

\subsection{Explicit fomulation}

Exact inverse kinematics computes the 6 joint values of the robotic arm with
respect to the input configuration variables:
\begin{equation}\label{eq:inverse-kinematics}
  \conf_{out} = f(\conf_{in}, rhs)
\end{equation}
Note that the input variables include an extra degree of freedom that is interpreted as an integer to select among the various solutions of the inverse kinematics.

\begin{figure}[h]
  \center
  \graphicspath{{./figures/}}
  \input{figures/staubli-TX90-schematic.pdf_tex}
  \caption{Kinematic diagram of the Stäubli TX2 series robots.}
  \label{fig:staubli-tx2}
\end{figure}
Figure~\ref{fig:staubli-tx2} represents the kinematic chain of Stäubli TX2 series robots.

To compute function $f$ in~(\ref{eq:inverse-kinematics}), we will determine the pose of the last robt joint in the base link of the robot and then use the kinematics of the robot to compute the joint values. Given the pose $^0M_h$ of the handle, we compute the desired pose $^0M_g$ of the gripper in the world frame from~(\ref{eq:implicit-formulation}):
\begin{align*}
  ^0M_g &= \;^0M_h rhs^{-1}\\
  ^0M_{bl}\;^{bl}M_a\;^aM_g &= \;^0M_h rhs^{-1} \\
  \;^{bl}M_a &= \;^0M_{bl}^{-1}\;^0M_h rhs^{-1} \;^aM_g^{-1}
\end{align*}
In the following computations, \texttt{joint a} becomes joint 6. We denote by:
\begin{itemize}
\item $T_{i-1,i}$ the homogeneous transformation matrix from frame $i-1$ to frame $i$,\\
\item $T_{0,1}$ the transformation matrix from base link to the first joint frame,\\
\item $T_{5,6}$ the transformation matrix from joint 5 frame to the gripper frame,\\
\end{itemize}
Homogeneous transformation matrix  $T_{i-1,i}$  from frame $i-1$  to frame $i$  :

\[
\begin{array}{cc}
T_{0,1} =
\begin{bmatrix}
  \cos(q_1) & -\sin(q_1) & 0 & 0 \\
  \sin(q_1) & \cos(q_1) & 0 & 0 \\
  0 & 0 & 1 & r_1 \\
  0 & 0 & 0 & 1 \\

\end{bmatrix}

&

T_{1,2} =
\begin{bmatrix}
  \cos(q_2) & 0 & \sin(q_2) & r_2 \\
  0 & 1 & 0 & 0 \\
  -\sin(q_2) & 0 & \cos(q_2) & 0 \\
  0 & 0 & 0 & 1 \\

\end{bmatrix}
\end{array}
\]

\[
\begin{array}{cc}
T_{2,3} =
\begin{bmatrix}
  \cos(q_3) & 0 & \sin(q_3) & 0 \\
  0 & 1 & 0 & r_3 \\
  -\sin(q_3) & 0 & \cos(q_3) & r_4 \\
  0 & 0 & 0 & 1 \\

\end{bmatrix}
&
T_{3,4} =
\begin{bmatrix}
  \cos(q_4) & -\sin(q_4) & 0 & 0 \\
  \sin(q_4) & \cos(q_4) & 0 & 0 \\
  0 & 0 & 1 & 0 \\
  0 & 0 & 0 & 1 \\

\end{bmatrix}
\end{array}
\]
\[
\begin{array}{cc}
T_{4,5} =
\begin{bmatrix}
  \cos(q_5) & 0 & \sin(q_5) & 0 \\
  0 & 1 & 0 & 0 \\
  -\sin(q_5) & 0 & \cos(q_5) & r_5 \\
  0 & 0 & 0 & 1 \\

\end{bmatrix}

&

T_{5,6} =
\begin{bmatrix}
  \cos(q_6) & -\sin(q_6) & 0 & 0 \\
  \sin(q_6) & \cos(q_6) & 0 & 0\\
  0 & 0 & 1 & r_6\\
  0 & 0 & 0 & 1\\

\end{bmatrix}
\end{array}
\]

Let us notice that the last three axes are concurrent. The concurrent point is the center of axis 5 and can be obtained by applying $T_{0,6}$ to vector $\left(0\ 0\ -r_6\right)$.
Calculation of  $T_{0,5}$ and $T_{3,6}$ are given below. We denote $sin(q_i)=s_i$  for each $i$ between 1 and 6.
\begin{align*}
T_{0,5} = T_{0,1} T_{1,2} T_{2,3} T_{3,4} T_{4,5} =
\begin{bmatrix}
  X & X & X &r_5c_1s_{2+3}-r_3s_1+r_4c_1s_2+r_2c_1\\
  X & X & X & r_5s_1s_{2+3}+r_3c_1+r_4s_1s_2+r_2s_1 \\
  X & X & X & r_5c_{2+3}+r_4c_2+r_1 \\
  0 & 0 & 0 & 1 \\
\end{bmatrix}
\end{align*}

\subsubsection*{Computation of $q_1$}

Let us rewrite
\begin{align*}
  X&=r_5c_1s_{2+3}-r_3s_1+r_4c_1s_2+r_2c_1\\
  Y&=r_5s_1s_{2+3}+r_3c_1+r_4s_1s_2+r_2s_1\\
  Z&=r_5c_{2+3}+r_4c_2+r_1
\end{align*}
the coordinates of the translation part of $T_{0,5}$. Importantly, this point is the intersection of the last 3 axes.
Given the geometry of the robot, we search $(X,Y)$ in the form:
\begin{equation}\label{eq:XY}
  \begin{bmatrix}
    c1 & -s1\\
    s_1 & c_1\\
  \end{bmatrix}
  \begin{bmatrix}
    x\\
    r_3\\
  \end{bmatrix}
  =
  \begin{bmatrix}
    c_1x-s_1r_3\\
    s_1x+c_1r_3\\
  \end{bmatrix}
  =
  \begin{bmatrix}
    X\\Y
  \end{bmatrix}
\end{equation}

Replacing $X$ and $Y$ by their expressions, we get:
\begin{align*}
  c_1x-\cancel{s_1r_3}&=r_5c_1s_{2+3}-\cancel{s_1r_3} +c_1s_2r_4+ c_1r_2\\
  s_1x+\cancel{c_1r_3}&=r_5s_1s_{2+3}+\cancel{c_1r_3}+s_1s_2r_4+s_1r_2\\
\end{align*}
we simplify by $c_1$ and $s_1$ :
\begin{equation}\label{eq:x}
  x=r_5s_{2+3}+s_2r_4+r_2
\end{equation}
We need to determine $q_1$ from these expressions. 
the $x$ and $y$ components of the translation part of $T_{0,5}$. Multiplying the first line of~(\ref{eq:XY}) by $s_1$ and the second line by $c_1$, we get
\begin{align*}
  s_1 c_1 x - s_1^2 r_3 &= s_1 X\\
  s_1 c_1 x + c_1^2 r_3 &= c_1 Y
\end{align*}
Substracting the second line to the first one, we get
$$
r_3 = c_1 Y - s_1 X.
$$
Let us express $(X,Y)$ in polar coordinates:
\begin{align}\label{eq:polar coordinates}
  X = \rho \cos\theta\\
  Y = \rho \sin\theta \\
  \label{eq:theta bounds}
  \rho \geq 0, -\pi \leq \theta \leq \pi.
\end{align}
We get
$$
r_3 = \rho(\cos q_1 \sin\theta - \sin q_1 \cos\theta) = \rho\sin(\theta - q_1).
$$
If $\rho < r_3$, then there is no solution. Otherwise,
$$
\sin(\theta - q_1) = \frac{r_3}{\rho}
$$
Then
$$
\theta - q_1 = \arcsin \frac{r_3}{\rho} + 2k\pi,\ \mbox{or}\ \theta - q_1 = \pi - \arcsin\frac{r_3}{\rho} + 2k\pi, k\in\mathbb{Z}.
$$
and
\begin{align*}
  q_1 &= \theta - \arcsin \frac{r_3}{\rho} + 2k\pi\ \mbox{or}\\
  q_1 &= \theta + \pi + \arcsin\frac{r_3}{\rho} + 2k\pi, k\in\mathbb{Z}.
\end{align*}

\paragraph{Possible solutions.}

\begin{align*}
  &0 < \arcsin \frac{r_3}{\rho} \leq \frac{\pi}{2}
\end{align*}
so from~(\ref{eq:theta bounds}),
\begin{align*}
  -\frac{3\pi}{2} &\leq \theta - \arcsin \frac{r_3}{\rho} \leq \pi\\
   0 & \leq \theta + \pi + \arcsin\frac{r_3}{\rho} \leq 2\pi + \frac{\pi}{2}
\end{align*}

As $-\pi < q_1 < \pi$, there is possibly 4 solutions:
\begin{align*}
  q_1 &= \theta - \arcsin \frac{r_3}{\rho} \\
  q_1 &= \theta - \arcsin \frac{r_3}{\rho} + 2\pi\\
  q_1 &= \theta + \pi + \arcsin\frac{r_3}{\rho} \\
  q_1 &= \theta - \pi + \arcsin\frac{r_3}{\rho}
\end{align*}

\subsubsection*{Computation of $q_3$}

\begin{figure}[h]
  \begin{center}
    \def\svgwidth {.6\linewidth}
    \graphicspath{{./figures/}}
    \input{figures/q2-q3.pdf_tex}
  \end{center}
  \caption{From the lengths of each side of triangle ABC, we can compute $\theta_3$}
  \label{fig:triangle}
\end{figure}

From $q_1$  by inverting~(\ref{eq:XY}), we get the value of $x$:
$$
x = c_1 X + s_1 Y.
$$
Using~(\ref{eq:x}), we can write
\begin{align}
  \label{eq:x:1}
  x &= r_5\sin(q_{2}+q_{3})+r_4\sin q_2+r_2 \\
  \label{eq:Z:1}
  Z &= r_5\cos(q_{2}+q_{3})+r_4\cos q_2+r_1
\end{align}
These equations define a triangle denoted ABC in Figure~\ref{fig:triangle}. Thus we can write
\begin{align*}
  \|\vec{AC}\|^2 &= \|\vec{AB}+\vec{BC}\|^2 = \|\vec{AB}\|^2 + \|\vec{BC}\|^2 + 2\vec{AB}.\vec{BC}\\
  (x-r_2)^2 + (Z-r_1)^2 &= r_4^2 + r_5^2 + 2r_4r_5\cos q_3 \\
  \cos q_3 &= \frac{(x-r_2)^2 + (Z-r_1)^2 - r_4^2 - r_5^2}{2r_4r_5}
\end{align*}
If the right hand side of the latter is bigger than 1, there is no solution. Otherwise
\begin{equation}\label{eq:q3}
  q_3 = \pm \arccos \frac{(x-r_2)^2 + (Z-r_1)^2 - r_4^2 - r_5^2}{2r_4r_5} + 2k\pi, k\in\mathbb{Z}
\end{equation}

\paragraph{Possible solutions.}

$$
0 \leq \arccos \frac{(x-r_2)^2 + (Z-r_1)^2 - r_4^2 - r_5^2}{2r_4r_5} \leq \pi
$$
and $-\pi < q_3 < \pi$. Thus, there are 2 solutions:
\begin{align*}
  q_3 &= + \arccos \frac{(x-r_2)^2 + (Z-r_1)^2 - r_4^2 - r_5^2}{2r_4r_5}\\
  q_3 &= - \arccos \frac{(x-r_2)^2 + (Z-r_1)^2 - r_4^2 - r_5^2}{2r_4r_5}
\end{align*}
\subsubsection*{Computation of $q_2$}

If we develop~(\ref{eq:x:1})-(\ref{eq:Z:1}), we get
\begin{align*}
  x - r_2 &= r_5(\sin q_2 \cos q_3 + \sin q_3 \cos q_2) + r_4 \sin q_2 \\
  Z -r_1 &= r_5(\cos q_2 \cos q_3 - \sin q_2 \sin q_3) + r_4\cos q_2\\
  x - r_2 &= r_5\sin q_3 \cos q_2  + (r_5\cos q_3 + r_4)\sin q_2\\
  Z -r_1 &= (r_5\cos q_3 + r_4)\cos q_2 - r_5\sin q_3\sin q_2
\end{align*}
Using Kramer's formula, we have
\begin{align*}
  \cos q_2 &= \frac{r_5\sin q_3(x-r_2) + (r_5\cos q_3 + r_4)(Z-r_1)}{r_4^2 + r_5^2 + 2r_4r_5\cos q_3}\\
  \sin q_2 &= \frac{(r_5\cos q_3 + r_4)(x-r_2)-r_5\sin q_3(Z-r_1)}{r_4^2 + r_5^2 + 2r_4r_5\cos q_3}
\end{align*}
As $-\pi < q_2 < \pi$, this system of equations has a unique solution.

\subsubsection*{Computation of $q_5$}

With the first 3 angles, we can compute the orientation matrix $R_{03}$ of axis 3:
$$
R_{0,3} =
\begin{bmatrix}
  c_1c_2c_3 - c_1s_2s_3 & -s_1 & c_1c_2s_3 + c_1s_2c_3\\
  s_1c_2c_3 - s_1s_2s_3 & c_1  & s_1c_2s_3 + s_1s_2c_3 \\
  -s_2c_3-c_2s_3        & 0   & -s_2s_3 + c_2c_3
\end{bmatrix}
$$
From this matrix, we can compute $R_{3,6}=R_{0,3}^T R_{0,6}$ where $R_{0,6}$ is the orientation of the
end effector that is given as input. Using the expressions of each axis rotation matrix given at the beginning if this section, we get an expression of $R_{3,6}$ with respect to the last three angles:
$$
R \triangleq R_{3,6}=
\begin{bmatrix}
  c_4c_5c_6-s_4s_6 & -c_4c_5s_6-s_4c_6 & c_4s_5 \\
  s_4c_5c_6+c_4s_6 & -s_4c_5s_6+c_4c_6 & s_4s_5 \\
  -s_5c_6 & s_5s_6 & c_5
\end{bmatrix}
$$
The coefficient of the last row and of the last column enables us to compute $q_5$:
\begin{align*}
  q_5 &= \pm \arccos R_{3\,3} + 2k\pi.
\end{align*}

\paragraph{Possible solutions.}

\begin{align*}
  0 &\leq \arccos R_{3\,3} \leq \pi
\end{align*}
and $-\pi < q_5 < \pi$. Thus, $q_5$ admits at most two solutions:
\begin{align*}
  q_5 &= \arccos R_{3\,3}\\
  q_5 &= -\arccos R_{3\,3}\\
\end{align*}

\subsubsection*{Computation of $q_4$}

If $s_5\not=0$, we get $q_4$ from  coefficients $R_{1\,3}$ and $R_{2\,3}$:
\begin{align*}
  q_4 &= \mbox{arctan2}(\frac{R_{2\,3}}{\sin q_5}, \frac{R_{1\,3}}{\sin q_5}) + 2k\pi,
\end{align*}

\paragraph{Possible solutions.}

\begin{align*}
  -\pi \leq \mbox{arctan2}(\frac{R_{2\,3}}{\sin q_5}, \frac{R_{1\,3}}{\sin q_5}) < \pi.
\end{align*}
As $-\frac{3\pi}{2} \leq q_4 \leq \frac{3\pi}{2}$, $q_4$ admits 3 solutions:
\begin{align*}
  q_4 &= \mbox{arctan2}(\frac{R_{2\,3}}{\sin q_5}, \frac{R_{1\,3}}{\sin q_5}) -2\pi \\
  q_4 &= \mbox{arctan2}(\frac{R_{2\,3}}{\sin q_5}, \frac{R_{1\,3}}{\sin q_5}) \\
  q_4 &= \mbox{arctan2}(\frac{R_{2\,3}}{\sin q_5}, \frac{R_{1\,3}}{\sin q_5}) + 2\pi
\end{align*}

\subsubsection*{Computation of $q_6$}

If $s_5\not=0$, $q_6$ from coefficients $R_{3\,1}$ and $R_{3\,2}$:
\begin{align*}
  q_6 &= \mbox{arctan2}(\frac{R_{3\,2}}{\sin q_5}, \frac{R_{3\,1}}{-\sin q_5}) + 2k\pi.
\end{align*}

\paragraph{Possible solutions.}

\begin{align*}
  -\pi \leq \mbox{arctan2}(\frac{R_{3\,2}}{\sin q_5}, \frac{R_{3\,1}}{-\sin q_5}) < \pi.
\end{align*}
As $-\frac{3\pi}{2} \leq q_6 \leq \frac{3\pi}{2}$, $q_6$ admits 3 solutions:
\begin{align*}
  q_6 &= \mbox{arctan2}(\frac{R_{3\,2}}{\sin q_5}, \frac{R_{3\,1}}{-\sin q_5}) -2\pi \\
  q_6 &= \mbox{arctan2}(\frac{R_{3\,2}}{\sin q_5}, \frac{R_{3\,1}}{-\sin q_5}) \\
  q_6 &= \mbox{arctan2}(\frac{R_{3\,2}}{\sin q_5}, \frac{R_{3\,1}}{-\sin q_5}) + 2\pi
\end{align*}

\subsubsection*{Encoding of the solutions}

To encode the solution, we use an extra degree of freedom that stores an integer. The table below
gathers the possible solutions for all joint angles

\begin{tabular}{|l|l|l|}
  \hline
  configuration variable & solutions & numbering \\
  \hline
  $q_1$ & $\theta - \arcsin \frac{r_3}{\rho}$ & $i_1=0$\\
   & $\theta - \arcsin \frac{r_3}{\rho} + 2\pi$ & $i_1=1$\\
   & $\theta + \pi + \arcsin\frac{r_3}{\rho}$& $i_1=2$ \\
   & $\theta - \pi + \arcsin\frac{r_3}{\rho}$& $i_1=3$\\
  \hline
  $q_3$ & $+ \arccos \frac{(x-r_2)^2 + (Z-r_1)^2 - r_4^2 - r_5^2}{2r_4r_5}$ & $i_3=0$\\
  &$- \arccos \frac{(x-r_2)^2 + (Z-r_1)^2 - r_4^2 - r_5^2}{2r_4r_5}$ & $i_3=1$ \\
  \hline
  $q_5$ & $\arccos R_{3\,3}$ & $i_5 =0$ \\
  & $-\arccos R_{3\,3}$ & $i_5=1$ \\
  \hline
  $q_4$ & $\mbox{arctan2}(\frac{R_{2\,3}}{\sin q_5}, \frac{R_{1\,3}}{\sin q_5}) -2\pi$ & $i_4=0$ \\
        & $\mbox{arctan2}(\frac{R_{2\,3}}{\sin q_5}, \frac{R_{1\,3}}{\sin q_5})$ & $i_4=1$\\
        & $\mbox{arctan2}(\frac{R_{2\,3}}{\sin q_5}, \frac{R_{1\,3}}{\sin q_5}) + 2\pi$ & $i_4=2$ \\
  \hline
  $q_6$ &$\mbox{arctan2}(\frac{R_{3\,2}}{\sin q_5}, \frac{R_{3\,1}}{-\sin q_5}) -2\pi$ & $i_6=0$ \\
        &$\mbox{arctan2}(\frac{R_{3\,2}}{\sin q_5}, \frac{R_{3\,1}}{-\sin q_5})$ & $i_6=1$ \\
  &$\mbox{arctan2}(\frac{R_{3\,2}}{\sin q_5}, \frac{R_{3\,1}}{-\sin q_5}) + 2\pi$ & $i_6=2$ \\
  \hline
\end{tabular}
The indices of each configuration variables are linked to the extra degree of freedom $i$ by the
following formula:
$$
i = i_1 + 4 i_3 + 8 i_5 + 16 i_4 + 48 i_6
$$
$i_1,i_3,i_5,i_4,i_6$ are retrieved from $i$ by computing successive Euclidean divisions.
\begin{itemize}
\item $i_1$ is the remainder of $i$ divided by 4,
\item $i_3$ is the remainder of $\frac{i-i_1}{4}$ divided by 8,...
\end{itemize}

\section{Jacobian}

In order to implement exact inverse kinematics as an explicit constraint, we need to compute the Jacobian of $f$. For that, let us consider a motion of the kinematic chain that keeps the gripper and handle in the same pose:
\begin{equation}\label{eq:jac1}
\forall t\in\reals,\ ^0M_b(t)\;^bM_h = \;^0M_r(t) \;^rM_a(t)\;^aM_g\;rhs
\end{equation}
Moreover
\begin{equation}\label{eq:jacobian arm}
\left(\begin{array}{c}
  \;^r\linvel_{a/r} \\ \;^r\omega_{a/r}
\end{array}\right) =
J_{out} \dot{\conf}_{out}
\end{equation}
where $J_{out}$ is the 6x6 matrix composed of the columns of Jacobian of \texttt{joint1} corresponding to the arm degrees of freedom. We denote respectively by $J_{out\;\linvel}$ and $J_{out\;\omega}$ the first 3 and the last 3 lines of this matrix.

Let us denote by
$$
^aM_{g'} = \;^aM_g rhs
$$

Using homogeneous matrix notation and derivating with respect to time, Equation~(\ref{eq:jac1}) can be written as $\forall t\in\reals$,
\begin{align}\label{eq:jac2}
  ^0M_b
  \left(\begin{array}{ll}[\;^b\omega_{b/0}]_{\times} & \;^b\linvel_{b/0} \\ 0&0\end{array}\right)\;^bM_h =&
    \;^0M_r\left(\begin{array}{ll}[\;^r\omega_{r/0}]_{\times} & \;^r\linvel_{r/0} \\ 0&0\end{array}\right) \;^rM_a\;^aM_{g'} \\
    & + \;^0M_r \;^rM_a\left(\begin{array}{ll}[\;^a\omega_{a/r}]_{\times} & \;^a\linvel_{a/r} \\ 0&0\end{array}\right)\;^aM_{g'} &
\end{align}
\begin{align*}
  \left(\begin{array}{ll}^0R_b[^b\omega_{b/0}]_{\times} & ^0R_b\;^b\linvel_{b/0} \\ 0&0\end{array}\right)\;^bM_h =&
    \left(\begin{array}{ll}^0R_r[^r\omega_{r/0}]_{\times} & ^0R_r\;^r\linvel_{r/0} \\ 0&0\end{array}\right) \;^rM_a\;^aM_{g'} \\
    & + \;^0M_r \left(\begin{array}{ll}^rR_a[^a\omega_{a/r}]_{\times} & ^rR_a\;^a\linvel_{a/r} \\ 0&0\end{array}\right)\;^aM_{g'} &
\end{align*}

%% \begin{align*}
%%   \left(\begin{array}{ll}^0R_b[^b\omega_{b/0}]_{\times} & ^0R_b\;^b\linvel_{b/0} \\ 0&0\end{array}\right)\left(\begin{array}{ll}\;^bR_h & \;^b\trans_h\\ 0&1\end{array}\right) =&
%%     \left(\begin{array}{ll}^0R_r[^r\omega_{r/0}]_{\times} & ^0R_r\;^r\linvel_{r/0} \\ 0&0\end{array}\right) \left(\begin{array}{ll}^rR_{g'} &\;^r\trans_{g'}\\0&1\end{array}\right) \\
%%     + \left(\begin{array}{ll}^0R_r &\;^0\trans_r\\0&1\end{array}\right) \left(\begin{array}{ll}^rR_a[^a\omega_{a/r}]_{\times} & ^rR_a\;^a\linvel_{a/r} \\ 0&0\end{array}\right)\left(\begin{array}{ll}^aR_{g'} &\;^a\trans_{g'}\\0&1\end{array}\right) &
%% \end{align*}
%% \begin{align*}
%%   \left(\begin{array}{ll}^0R_b[^b\omega_{b/0}]_{\times}\;^bR_h & ^0R_b[^b\omega_{b/0}]_{\times}\;^b\trans_h + ^0R_b\;^b\linvel_{b/0} \\ 0&0\end{array}\right) =
%%     \left(\begin{array}{ll}^0R_r[^r\omega_{r/0}]_{\times}\;^rR_{g'} & ^0R_r[^r\omega_{r/0}]_{\times}\;^r\trans_{g'} + ^0R_r\;^r\linvel_{r/0} \\ 0&0\end{array}\right) \\
%%       + \left(\begin{array}{ll}^0R_r\;^rR_a[^a\omega_{a/r}]_{\times} & ^0R_r\;^rR_a\;^a\linvel_{a/r} \\ 0&0\end{array}\right)\left(\begin{array}{ll}^aR_{g'} &\;^a\trans_{g'}\\0&1\end{array}\right) &
%% \end{align*}

\begin{align*}
  \left(\begin{array}{ll}^0R_b[^b\omega_{b/0}]_{\times}\;^bR_h & ^0R_b[^b\omega_{b/0}]_{\times}\;^b\trans_h + ^0R_b\;^b\linvel_{b/0} \\ 0&0\end{array}\right) =
    \left(\begin{array}{ll}^0R_r[^r\omega_{r/0}]_{\times}\;^rR_{g'} & ^0R_r[^r\omega_{r/0}]_{\times}\;^r\trans_{g'} + ^0R_r\;^r\linvel_{r/0} \\ 0&0\end{array}\right) \\
      + \left(\begin{array}{ll}^0R_r\;^rR_a[^a\omega_{a/r}]_{\times}\;^aR_{g'} &
        ^0R_r\;^rR_a[^a\omega_{a/r}]_{\times}\;^a\trans_{g'}  + ^0R_r\;^rR_a\;^a\linvel_{a/r}\\ 0&0\end{array}\right)&
\end{align*}
Extracting the upper blocks of this matrix equality, we get
\begin{align*}
  ^0R_b[^b\omega_{b/0}]_{\times}\;^bR_h &= ^0R_r[^r\omega_{r/0}]_{\times}\;^rR_{g'} + ^0R_r\;^rR_a[^a\omega_{a/r}]_{\times}\;^aR_{g'} \\
  ^0R_b[^b\omega_{b/0}]_{\times}\;^b\trans_h + ^0R_b\;^b\linvel_{b/0} &=
  ^0R_r[^r\omega_{r/0}]_{\times}\;^r\trans_{g'} + ^0R_r\;^r\linvel_{r/0} + ^0R_r\;^rR_a[^a\omega_{a/r}]_{\times}\;^a\trans_{g'}  + ^0R_r\;^rR_a\;^a\linvel_{a/r} \\
  ^0R_b[^b\omega_{b/0}]_{\times}\;^bR_h &= ^0R_r[^r\omega_{r/0}]_{\times}\;^rR_{g'} + ^0R_a[^a\omega_{a/r}]_{\times}\;^aR_{g'} \\
  ^0R_b[^b\omega_{b/0}]_{\times}\;^b\trans_h + ^0R_b\;^b\linvel_{b/0} &=
  ^0R_r[^r\omega_{r/0}]_{\times}\;^r\trans_{g'} + ^0R_r\;^r\linvel_{r/0} + ^0R_a[^a\omega_{a/r}]_{\times}\;^a\trans_{g'}  + ^0R_a\;^a\linvel_{a/r} \\
  [^0\omega_{b/0}]_{\times}\;^0R_h &= [^0\omega_{r/0}]_{\times}\;^0R_{g'} + [^0\omega_{a/r}]_{\times}\;^0R_{g'} \\
  ^0R_b[^b\omega_{b/0}]_{\times}\;^b\trans_h + ^0R_b\;^b\linvel_{b/0} &=
  ^0R_r[^r\omega_{r/0}]_{\times}\;^r\trans_{g'} + ^0R_r\;^r\linvel_{r/0} + ^0R_a[^a\omega_{a/r}]_{\times}\;^a\trans_{g'}  + ^0R_a\;^a\linvel_{a/r} \\
\end{align*}
As $\;^0R_h = \;^0R_{g'}$ all along the motion,
\begin{align*}
  ^0\omega_{b/0} &= \;^0\omega_{r/0} + \;^0\omega_{a/r}\\
  -^0R_b[\;^b\trans_h]_{\times}\;^b\omega_{b/0} + ^0R_b\;^b\linvel_{b/0} &=
  -^0R_r[\;^r\trans_{g'}]_{\times}\;^r\omega_{r/0} + ^0R_r\;^r\linvel_{r/0} - ^0R_a[\;^a\trans_{g'}]_{\times}\;^a\omega_{a/r} + \;^0R_a\;^a\linvel_{a/r} \\
  ^0R_b\;^b\omega_{b/0} &= \;^0R_r\;^r\omega_{r/0} + \;^0R_r\;^r\omega_{a/r}\\
  -^0R_b[\;^b\trans_h]_{\times}\;^b\omega_{b/0} + ^0R_b\;^b\linvel_{b/0} &=
  -^0R_r[\;^r\trans_{g'}]_{\times}\;^r\omega_{r/0} + ^0R_r\;^r\linvel_{r/0} - ^0R_a[\;^a\trans_{g'}]_{\times} \;^aR_{r}\;^r\omega_{a/r} + \;^0R_a\;^a\linvel_{a/r}
\end{align*}
Using~(\ref{eq:jacobian arm}), we can write
\begin{align*}
  ^r\omega_{a/r} =& \;^rR_b\;^b\omega_{b/0} - \;^r\omega_{r/0}\\
  -^0R_b[\;^b\trans_h]_{\times}\;^b\omega_{b/0} + \;^0R_b\;^b\linvel_{b/0} =&
  -\;^0R_r[\;^r\trans_{g'}]_{\times}\;^r\omega_{r/0} + \;^0R_r\;^r\linvel_{r/0} \\ &-\;^0R_a[\;^a\trans_{g'}]_{\times} (\;^aR_b\;^b\omega_{b/0} - \;^aR_{r}\;^r\omega_{r/0}) + \;^0R_a\;^a\linvel_{a/r}
\end{align*}
\begin{align*}
  ^a\linvel_{a/r} =& \;^aR_0\left(
  -\;^0R_b[\;^b\trans_h]_{\times}\;^b\omega_{b/0} + \;^0R_b\;^b\linvel_{b/0}
  +\;^0R_r[\;^r\trans_{g'}]_{\times}\;^r\omega_{r/0} - \;^0R_r\;^r\linvel_{r/0} \right.\\
  &\left.+\;^0R_a[\;^a\trans_{g'}]_{\times} (\;^aR_b\;^b\omega_{b/0} - \;^aR_{r}\;^r\omega_{r/0})  \right)\\
  ^r\omega_{a/r} =& \;^rR_b\;^b\omega_{b/0} - \;^r\omega_{r/0}
\end{align*}
\begin{align}
  ^a\linvel_{a/r} =&
  -\;^aR_b[\;^b\trans_h]_{\times}\;^b\omega_{b/0} + \;^aR_b\;^b\linvel_{b/0}
  +\;^aR_r[\;^r\trans_{g'}]_{\times}\;^r\omega_{r/0} - \;^aR_r\;^r\linvel_{r/0}\\
  \label{eq:jac31}
  &+\;^0R_a[\;^a\trans_{g'}]_{\times} (\;^aR_b\;^b\omega_{b/0} - \;^aR_{r}\;^r\omega_{r/0})\\
  \label{eq:jac32}
  ^r\omega_{a/r} =& \;^rR_b\;^b\omega_{b/0} - \;^r\omega_{r/0}
\end{align}
We denote
\begin{itemize}
\item $J_{b\;in}$ the columns of the Jacobian of \texttt{joint2} corresponding to the input variables,
\item $J_{b\;in}^{\linvel}$, $J_{b\;in}^{\omega}$, respectively the first 3 and last 3 lines of the latter,
\item $J_{r\;in}$ the columns of the Jacobian of \texttt{root} corresponding to the input variables,
\item $J_{r\;in}^{\linvel}$, $J_{r\;in}^{\omega}$, respectively the first 3 and last 3 lines of the latter,
\end{itemize}
With this notation, (\ref{eq:jac31}-\ref{eq:jac32}) become
\begin{align*}
  ^a\linvel_{a/r} =&
  \left(-\;^aR_b[\;^b\trans_h]_{\times}J_{b\;in}^{\omega} + \;^aR_bJ_{b\;in}^{\linvel}
  +\;^aR_r[\;^r\trans_{g'}]_{\times}J_{r\;in}^{\omega} - \;^aR_rJ_{r\;in}^{\linvel}\right.\\
  &\left.+\;^0R_a[\;^a\trans_{g'}]_{\times} (\;^aR_bJ_{b\;in}^{\omega} - \;^aR_{r}J_{r\;in}^{\omega})\right)\dot{\conf}_{in}\\
  ^r\omega_{a/r} =& \left(\;^rR_bJ_{b\;in}^{\omega} - J_{r\;in}^{\omega}\right)\dot{\conf}_{in}
\end{align*}
Let $J$ be the 6 matrix the first 3 lines of which are
$$
^aR_b(-[\;^b\trans_h]_{\times}J_{b\;in}^{\omega} + J_{b\;in}^{\linvel})
+\;^aR_r([\;^r\trans_{g'}]_{\times}J_{r\;in}^{\omega} - J_{r\;in}^{\linvel}) + \;^0R_a[\;^a\trans_{g'}]_{\times} (\;^aR_bJ_{b\;in}^{\omega} - \;^aR_{r}J_{r\;in}^{\omega})
$$
and the last 3 lines of which are
$$
^rR_bJ_{b\;in}^{\omega} - J_{r\;in}^{\omega}
$$
Using~(\ref{eq:jacobian arm}), we can write
$$
\dot{\conf}_{out} = J_{out}^{-1}J \dot{\conf}_{in} \;\mbox{ and }\; \frac{\partial f}{\partial \conf_{in}} = J_{out}^{-1}J
$$
\bibliographystyle{plain}
\bibliography{inverse-kinematics}

\end{document}
